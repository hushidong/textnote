% !Mode:: "TeX:UTF-8"
\documentclass{article}%[11pt]
\usepackage{ctex}
\usepackage{lipsum}
\usepackage{textnote}


\begin{document}

当然,除了\textnote{唐诗, 春晓, 作者: 孟浩然}小脚盆和棒子猴子之外,还有东南沿海之外的菲佣国;还有世界屋脊那边的三哥国,三哥国和猴子国之间还有一个竹楼国、一个佛陀国、一个咖喱国;还有跟三哥国\textnote{唐诗, 静夜思, 作者: 李白}互相看不上眼的小巴国;小巴国北面还有好几个骆驼国……这么多大大小小的国家崽子,环形拱卫着土鳖国,名为邦国\textnote{唐诗, 将进酒, 作者: 李白},实为外藩,每隔几年都要向土鳖交保护费的说……背景如此复杂,精彩吧?

\lipsum[1-3]

当然了,那时候的\textnote{note contents A}土鳖可从来不承认自己是土鳖,它有一个更加响亮的名字:龙的传人!听听,龙的传人!有谁知道龙是神马东西?不知道木关系,三国时候的曹大白脸有标注:龙能大能小,能升能隐,大则兴云吐雾,小则隐介藏形。不是和老花一样,有点懵圈?\textnote{note contents B}好吧,通俗点说,龙这家伙虎须鬣尾,身长若蛇,有鳞若鱼,有角仿鹿,有爪似鹰,能走,亦能飞,能倒水,能大能小,能隐能现,能翻江倒海,吞风吐雾,兴云降雨。还是不懂?木关系,咱们直接说龙长的啥样:龙的形状是鹿的角,牛的耳朵,驼的头,兔的眼,蛇的颈,蜃的腹,鱼的鳞,鹿的脚掌,鹰的爪子。这就是有名的龙有九似,它来源于土鳖最古老的图腾蛇。

\lipsum[1-2]

当然了,那时候的\textnote{note contents A}土鳖可从来不承认自己是土鳖,它有一个更加响亮的名字:龙的传人!听听,龙的传人!有谁知道龙是神马东西?不知道木关系,三国时候的曹大白脸有标注:龙能大能小,能升能隐,大则兴云吐雾,小则隐介藏形。不是和老花一样,有点懵圈?

\lipsum[3]

\end{document}


