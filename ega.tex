% !Mode:: "TeX:UTF-8"
\documentclass{article}%[11pt]
\usepackage{ctex}
\usepackage[left=3cm,right=2cm]{geometry}%[showframe,showcrop]

\usepackage{textnote}

\usepackage[type=none,showframe=true]{fgruler}
%[unit=cm,type=lowerleft,showframe=true,hshift=3cm,vshift=2cm]
\rulerparams{}{}{red}{}{0.5pt}
\fgrulerdefnum{}\fgrulercaptioncm{}%fgruler加数字后,导致基线对齐出现问题,所以这里去掉


\newlength{\skipheadrule}
\deflength{\skipheadrule}{3.5pt}
\newlength{\skipfootrule}
\deflength{\skipfootrule}{5.5pt}
\newlength{\ruletotalen}
\deflength{\ruletotalen}{\textheight}
\newlength{\ruleraised}
\deflength{\ruleraised}{\headsep+\textheight}
\usepackage{fancyhdr}
\fancyhead[L]{%
\raisebox{-\skipheadrule}{%
\raisebox{-\headsep}[0pt][0pt]{\makebox[0pt][l]{\ruler{rightup}{\linewidth}}}%
\raisebox{-\ruleraised}[0pt][0pt]{\makebox[0pt][r]{\ruler{upleft}{\ruletotalen}}}%
}HEAD LEFT%
}
\fancyhead[R]{%
HEAD RIGHT%
\raisebox{-\skipheadrule}{%
\hfill\makebox[0pt][l]{\raisebox{-\ruleraised}[0pt][0pt]{\ruler{downright}{\ruletotalen}}\hss}%
}}
\fancyhead[C]{%
HEAD CENTER
}
\fancyfoot[L]{%
\raisebox{-\skipfootrule}{%
\raisebox{\footskip}[0pt][0pt]{\makebox[0pt][l]{\ruler{rightdown}{\linewidth}}}
}FOOT LEFT
}
\fancyfoot[C]{%
FOOT CENTER \thepage
}
\fancyfoot[R]{%
FOOT RIGHT
}
\renewcommand{\headrulewidth}{0.4pt}
\renewcommand{\footrulewidth}{0.4pt}

\pagestyle{fancy}


\begin{document}

%\show\output

\drawgrid{\linewidth}%
当然,除了\textnote{唐诗, 春晓, 作者: 孟浩然}小脚盆和棒子猴子之外,还有东南沿海之外的菲佣国;还有世界屋脊那边的三哥国,三哥国和猴子国之间还有一个竹楼国、一个佛陀国、一个咖喱国;还有跟三哥国\textnote{唐诗, 静夜思, 作者: 李白}互相看不上眼的小巴国;小巴国北面还有好几个骆驼国……这么多大大小小的国家崽子,环形拱卫着土鳖国,名为邦国\textnote{唐诗, 将进酒, 作者: 李白},实为外藩,每隔几年都要向土鳖交保护费的说……背景如此复杂,精彩吧?


%4.3cm有问题
\begin{figure}[b]
  \centering
  \fbox{\parbox[c][3.1cm][c]{8cm}{\centering float example}}
  \caption{float example}\label{fig:floateg}
\end{figure}

\newlength{\zitigao}
\deflength{\zitigao}{10.5bp}

小脚盆国还很年轻的时候,是最恭顺的孩子,心中有多大的不满也不说,美其名曰忍术!\footnote{这也不怪老土鳖瞎眼,任谁看了都会可怜}
这也不怪老土鳖瞎眼,任谁看了都会可怜,小国寡民的,悬孤岛于海外,感受不到中原帝国的风采也就罢了,最主要的是后自己家的几个小岛,都是海底火山冒出水面的礁石,谁也搞不清楚这个超级大炮仗神马时候炸响……小脚盆国人的苦逼之甚,由此。\the\baselineskip\the\zitigao

那时候%
%\testparnote{唐诗 春晓\\
%作者:孟浩然\\
%春眠不觉晓,处处闻啼鸟。\\
%夜来风雨声,花落知多少。}%
的小脚盆为了出人头地,还是很下了一番苦功夫努力的
\footnote{好好学习有木有?}。好好学习有木有?天天向上有木有?土鳖看了自然欣慰,私下里免不了传授了一些秘籍,盼望着这孩子长大了能出息,能出人头地,能挺直了腰杆子做人……钢铁是怎么炼成的我不知道,白眼狼一般就都是这么炼成滴。

事实上,天不遂人愿的事情多了去了,大侠杨过都说了:“不如意事常八九,可以人言无二三啊。”好吧,闲言少叙,下面我们隆重介绍猴子家族……木有错,是家族,不是只有一个,而是一口气搞出来好几个猴子国,有越猴,马来猴,印尼猴……等等等等吧,反正一大堆猴子国,面积说大不大,一般抵得上土鳖国一个到两个行省的规模,人口说少不少,几百万到几千万个别上亿还是有滴,猴子从来不搞神马计划生育,所以以上的数据说起来有点儿气人,就不说了。

猴儿国们都在土鳖国的南方,向来被土鳖当成自己家的后院儿,不许别人跑来乱串门子。说完了两个比较有名的属国,这周遭的其他国家也不能不表表,谁让这个世界这么精彩来着?当然也可以说这个世界怎么那么他奶奶的不让人顺心呢,如果这个世界只有土鳖一家,岂不是啥问题也没有了?说完了两个比较有名的属国,这周遭的其他国家也不能不表表,谁让这个世界这么精彩来着?当然也可以说这个世界怎么那1234唐诗么他奶奶的不让人顺心呢,如果这个世界只有土鳖一家,岂不是啥问题也没有了?说完了两个比较有名的属国,\textnote{唐诗, 春望, 作者: 杜甫}这周遭的其他国家也不能不表表,谁让这个世界这么精彩来着?当然也可以说这个世界怎么那么他奶奶的不让人顺心呢,如果这个世界只有土鳖一家,岂不是啥问题也没有了?说完了两个比较有名的属国,这周遭的其他国家也不能不表表,谁让这个世界这么精彩来着?当然也可以说这个世界怎么那么他奶奶的不让人顺心呢,如果这个世界只有土鳖一家,岂不是啥问题也没有了?

说完了两个比较有名的属国,这周遭的其他国家也不能不表表,谁让这个世界这么精彩来着?当然也可以说这个世界怎么那么他奶奶的不让人顺心呢,如果这个世界只有土鳖一家,岂不是啥问题也没有了?说完了两个比较有名的属国,这周遭的其他国家也不能不表表,谁让这个世界这么精彩来着?当然也可以说这个世界怎么那么他奶奶的不让人顺心呢,如果这个世界只有土鳖一家,岂不是啥问题也没有了?

问题是,古代人对客观世界的认知,主要还是靠眼睛和耳朵,太过于宏大的东西超出了人类的理解能力范围,所以土鳖国的老祖宗们,一直以为天是圆的,地是方的,到了海边就以为到了世界尽头,从来没有想过海的那一边也有美女可以搭讪。

\linerightnote{唐诗 春晓\\
作者:孟浩然\\
春眠不觉晓,处处闻啼鸟。\\
夜来风雨声,花落知多少。}\parshape=1 0pt 0.5\linewidth
土鳖的历代大佬们,有值得骄傲的本钱:地盘占了一大块,而且还是最好的一大块,具体怎么个好法以后再说,先说说这块地盘有多大。往东,没的说,一直占到了大海边;往南,屏障在南方的猴国们都不大,过了猴国再往南看,又是茫茫大海;

\lineleftnote{唐诗 春晓\\
作者:孟浩然\\
春眠不觉晓,处处闻啼鸟。\\
夜来风雨声,花落知多少。}\parshape=1 0.5\linewidth 0.5\linewidth
往西,万丈高山平地起,高的山腰往上都不长草,而且绵延数百上千里,飞鸟难越,人迹罕至,指望着骑马打过去,是木有意义滴;最后就是北边,北边的游牧民族曾经给土鳖带来了很大的威胁。不过那都是老黄历了,游牧民族被收服的被收服,被打跑的被打跑,空出来好大一片草原。

\linerightnote{唐诗 春晓\\
作者:孟浩然\\
春眠不觉晓,处处闻啼鸟。\\
夜来风雨声,花落知多少。}\parshape=5 0pt 0.5\linewidth 0pt 0.5\linewidth 0pt 0.5\linewidth 0pt 0.5\linewidth 0pt \linewidth
当然了,那时候的土鳖可从来不承认自己是土鳖,它有一个更加响亮的名字:龙的传人!听听,龙的传人!有谁知道龙是神马东西?不知道木关系,三国时候的曹大白脸有标注:龙能大能小,能升能隐,大则兴云吐雾,小则隐介藏形。不是和老花一样,有点懵圈?好吧,通俗点说,龙这家伙虎须鬣尾,身长若蛇,有鳞若鱼,有角仿鹿,有爪似鹰,能走,亦能飞,能倒水,能大能小,能隐能现,能翻江倒海,吞风吐雾,兴云降雨。还是不懂?木关系,咱们直接说龙长的啥样:龙的形状是鹿的角,牛的耳朵,驼的头,兔的眼,蛇的颈,蜃的腹,鱼的鳞,鹿的脚掌,鹰的爪子。这就是有名的龙有九似,它来源于土鳖最古老的图腾蛇。

%\csuse{restvsize1}
%\csuse{restvsize2}
%\csuse{restvsize3}
%\csuse{restvsize4}
%
%\iftoggle{auxinfoused}{
%auxinfousedis true
%\deflength{\notevsize}{20pt}
%\parrestvsize{1}{\notevsize}\relax\the\matchvsize
%}{auxinfousedis false}

%\the\matchvsize
%\the\matchvsize


\end{document}


